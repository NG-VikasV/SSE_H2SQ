\documentclass[aps,prd,twocolumn,superscriptaddress,showpacs,floatfix]{revtex4-1}
\usepackage[utf8]{inputenc}
\usepackage[T1]{fontenc}
\usepackage{textcomp}
\usepackage{graphicx}
\usepackage[export]{adjustbox}
%\usepackage{caption}
\usepackage[export]{adjustbox}
\usepackage{amsmath,amsthm,amssymb}
\usepackage[caption=false]{subfig}
\usepackage{lmodern}
\DeclareUnicodeCharacter{B0}{\textdegree}
\usepackage{placeins}
%\usepackage{subfigure}
%\usepackage{caption=false}{subfig}
%\usepackage{subcaption}
\usepackage{amstext}
\usepackage{lipsum}
\usepackage{enumitem}
\usepackage{mwe}
\usepackage{mathtools}
\usepackage{afterpage}
\usepackage{epsfig}
\usepackage{epstopdf} %converting to PDF
\setcitestyle{square}
%\usepackage[backend=biber,style=numeric,sorting=none]{biblatex}
%\usepackage{babel}													%	remove from the final version
%\usepackage{filecontents}
\usepackage[colorlinks,citecolor=blue,filecolor=blue,linkcolor=blue,urlcolor=blue]{hyperref}
\newcommand{\mandal}[1]{\textcolor{red}{#1}}
\newcommand{\vikas}[1]{\textcolor{blue}{#1}}
\newcommand{\nandy}[1]{\textcolor{magenta}{#1}}
%\usepackage[colorlinks=true,linkcolor=red]{hyperref}
\captionsetup[subfigure]{justification=justified,singlelinecheck=false}



\begin{document}
\title{Quantum Monte Carlo analysis of role of interlayer coupling in an organic squaric acid crystal}
\author{Vikas Vijigiri}
\email{vikasvikki@iopb.res.in}
\affiliation{Institute of Physics, Bhubaneswar-751005, Orissa, India}
\affiliation{Homi Bhabha National Institute, Mumbai - 400 094, Maharashtra, India}
\author{Ashis K. Nandy}\email{aknandy@niser.ac.in}\affiliation{School of Physical Sciences, National Institute of Science Education and Research, Jatni, 752050, Odisha, India}
\affiliation{Homi Bhabha National Institute, Mumbai - 400 094, Maharashtra, India}
\author{Saptarshi Mandal }\email{saptarshi@iopb.res.in}\affiliation{Institute of Physics, Bhubaneswar-751005, Orissa, India}
\affiliation{Homi Bhabha National Institute, Mumbai - 400 094, Maharashtra, India}


\begin{abstract}
%We explore the zero-temperature phase diagram of the three-dimensional squaric acid crystal using discrete-time(Suzuki-Trotter decomposition) path integral quantum Monte Carlo technique. We consider a model Hamiltonian with the leading term consisting of a four-spin interacting term, $J_0$, within the plaquette, which along with the Intramolecular coupling term, $J_1$, conserves the ice-rules. Here, we extend the model to include the interlayer exchange coupling, $J_3$, to suit the realistic nature of the material. This work aims to characterize the phases and their corresponding critical boundaries using pseudo-spin formalism.
%We perform our calculations for a generic interlayer coupling with the nature of the interaction is taken at random. Our analysis on order parameter($P,\rho$) and the imaginary-time correlation function $c_\tau$ suggests the existence of an intermediate state with two distinct phases co-existing together. One being the ferroelectric global order and the other a deconfined phase with algebraic correlations.  As a special case, we also consider a uniform (anti)ferromagnetic coupling and observe that the material only gets more stabilized, as the phase boundary now starts shifting towards the right. Finally, we conclude our results by drawing the critical boundary with interlayer coupling as a function of the external transverse field $K_x$ both for random and uniform versions. We also give the results of dynamic structure factor $s^{xx}$ calculated within the Linear-spin wave regime for the 2D version and discuss the relevance of it in brief.

We introduce a three-dimensional pseudo-spin model for  squaric acid crystal ($\text{H}_2\text{SQ}$) which is obtained by  considering an Ising type interlayer coupling ($J_3$) between the two-dimensional layers \cite{vijigiri2018classical} of the squaric acid crystal. To explore the ice-rule dominated physics and its implications, we consider three types of interlayer coupling, i.e, ferroelectric, antiferroelectric and disordered. Firstly, within the spin-wave approximation we find that the spectrum and the spin wave velocity does not depend on the sign of interlayer coupling. Instead, the spin-wave velocity is isotropic and proportional to the magnitude of $J_3$. In the quantum regime, we analyze the model using a Suzuki-Trotter quantum monte carlo method and evaluate the imaginary-time correlation function, susceptibility and specific heat in respective parameter space. The phase diagram as a function of temperature (or external field) shows that the intermediate correlated phase with ice-rule dominated physics survives locally and gets extended for higher temperatures for FM interlayer coupling. Where as for AFM or disordered interlayer coupling the intermediate phase is qualitatively similar to individual two dimensional layer. This points out that strong quantum fluctuations for AFM or disordered coupling weakens the effect of interlayer coupling. Moreover, the transition temperature is decreased upon increasing in interlayer strength for AFM and disordered case, vice-versa for FM case. Finally, we discuss possible experimental ramifications of our analysis.
\end{abstract}


\date{\today}

\pacs{
77.80.-e,
75.10.Jm,
75.40.Mg,	
77.84.Fa
%71.10.Pm 	
%03.65.Ud 	
%03.65.Vf 
%03.67.Mn
}
\maketitle
\section{Introduction}
\label{intro}
\indent 
For a long time past, squaric acid ($\text{H}_2\text{C}_4\text{O}_4$, also known as $\text{H}_2\text{SQ}$) has been intensely investigated  exploring  various  aspects arising from the interacting hydrogen-bonds (H-bonds). The interacting H-bonds are responsible for a wide range of phenomena observed in nature such as proton transfer processes in enzyme catalysis~\cite{mathias2007structures}, concerted proton transfers in membrane water channels~\cite{brewer2001formation,dellago2003proton} and cooperative proton tunneling in water hexamers~\cite{chen2013nature,bove2009anomalous}. Recently, hydrogen-bonded systems are being looked into plausible technological devices which exploit the H-bonds to induce electrically ordered quantum states~\cite{lengyel2019}. There have been extensive studies looking into its strong electronic (anti) ferroelectric polarizability~\cite{fjaer1980light,petersson1980phase,horiuchi2018strong,dolin2011quantum,horiuchi2008organic,horiuchi2020hydrogen}. With the recent realization of room-temperature ferroelectricity observed in one such H-bonded ferroelectrics~\cite{horiuchi2005ferroelectricity,horiuchi2012above}, there seems to be a strong possibility for future applications. Apart from the ferroelectric properties, hydrogen-bonded systems could also be used in detecting confinement to deconfinement transition~\cite{huang2017deconfinement}. Though experimentally it is maybe a challenging task to detect the deconfined charges~\cite{huang2017deconfinement} yet it would validate certain lattice gauge theories~\cite{chern2014gauge}. \\
\indent An interesting aspect of hydrogen-bonded systems is that the hydrogen ions (or protons) satisfy local constraints known as the \textit{ice-rules}. Depending on the details of the electronic structure of the material, the number of protons engaged in the \textit{ice-rules} vary. For example, for materials (such as squaric acid) with fundamental unit as a square plaquette there are two protons that are nearer and two protons farther on any given plaquette in an ice-rule state. From these ice-rules and the structural phase transition perspective, among various H-bond systems the squaric acid bears many similarities with water ice~\cite{benton2016classical}. However, the significance of squaric acid comes from the simplicity of it's quasi-2D nature and also from the \textit{squaric} geometry of $\text{C}_4\text{O}_4$ molecules.\\% And also for the three-dimensional structure of hexagonal water ice ($I_h$) similar to the squaric acid which exists as weakly coupled quasi-two-dimensional layers. \\
%\indent The crystal structure of $\text{H}_2\text{SQ}$ consists of quasi-2D layers stacked one above the other with each alternating layer shear displaced slightly along the horizontal direction (see Fig.~\ref{lattice}(a)). It is known that despite its three-dimensional structure, the material is known to manifest its quasi-two-dimensional behavior. Specifically, low values of the critical exponent for order parameter ($f_i$= 0.17 and 0.14) obtained respectively from measurements of the optical birefringence~\cite{semmingsen1974structural} and of the neutron scattering~\cite{WANGY1974}, were found to have a characteristic of a two-dimensional transition. This poses a question as to how the layers have been stacked antiferroelectrically and what might be the role of interlayer coupling. However, from the experiments of \textit{Nakashima et al}~\cite{nakashima1976raman}, we know that the interlayer interaction acts only between the hydrogen bonds of adjacent layers and is localized in the region near the H-bonds. Following this, a theory has been put forth suggesting that the interlayer coupling may be the reason for high transition temperatures, $T_c$, and for a phase transition to happen~\cite{maier1980phenomenological}.\\
\indent It may be noted that various theoretical works were mostly limited to the quasi two-dimensional nature of squaric acid~\cite{ishizuka2011quantum,chern2014gauge,vijigiri2018classical,huang2017deconfinement,ishibashi2018computational,matsushita1982cluster}. Early theories based on order-disorder transition coupled with lattice (phonon) degrees of freedom motivated in explaining the (anti)ferroelectric order for a quasi-2D layer. This was supported by the findings where low values of the critical exponent for an order parameter ($f_i$= 0.17 and 0.14) were observed using optical birefringence~\cite{semmingsen1974structural} and  neutron scattering~\cite{WANGY1974} experiments. It is also known that the nearest neighbour two-dimensional planes are found to be (anti) ferroelectrically stacked~\cite{hollander1977molecular}. Although this indicates that an interlayer coupling between the two dimensional planes must exist and be weak, yet the nature of interaction to our knowledge is still unknown.  The squaric acid exhibits a characteristic phase transition from low temperature antiferroelectric phase to a high temperature paraelectric phase of first-order. Unlike the other H-bonded counterparts like $\text{KH}_2\text{PO}_4$, the transition temperature $T_c$ is unusually high pointing towards a possible role of interlayer coupling. This therefore, necessitates to consider an interlayer coupling  between adjacent layers for a qualitative understanding of transition temperatures versus interlayer coupling. And thus from a purely theoretical (adventurous) point of view, the study of three dimensional spin models become very important in its own right. Motivated by these, we here introduce and investigate a three dimensional pseudo-spin models applicable for $\text{H}_2\text{SQ}$. \\
\indent In our endeavour to study the three dimensional characteristics of proton dynamics relevant to $\text{H}_2\text{SQ}$, we extend the two-dimensional model studies~\cite{chern2014gauge, vijigiri2018classical} to three-dimensions by considering a nearest neighbour interlayer interaction of Ising type.
In the absence of first principle study, we consider all possible types of interlayer coupling such as ferroelectric (FM), antiferroelectric (AFM) and disordered type (where some bonds are randomly chosen ferromagnetic ($J_3 < 0$) and the rest to be antiferromagnetic ($ J_3 > 0$)).  We find that various phases such as long-range  antiferroelectric phase, quantum-liquid-like intermediate phase that are found in 2D-cases \cite{vijigiri2020dipole,ishizuka2011quantum} are sensitive in different ways in each of the different type of interlayer coupling. Although there is no distinction between the AFM and FM type coupling there is a significance deviation of physical quantities when the coupling is of disordered type. Briefly, our quantum Monte Carlo results show that the extent which the long ranged antiferroelectric phase extends in temperature is qualitatively same for AFM and FM type and different for disordered type interlayer coupling. In AFM and FM cases, the intermediate quantum-liquid like state vanishes slowly as when the strength ogsurvives for small values and vanishes after certain threshold interlayer coupling. As the interlayer coupling is increased the phase transition from AFE phase to quantum liquid-like state transforms to a phase transition from AFE to conventional paraelectric phase. Interestingly, our semi-classical analysis using spin-wave approximations corroborate the findings from QMC where the spin-wave spectrum is shown to be independent of the sign (type) of the interlayer coupling. In both the cases the the spin-wave velocity acquires a finite value proportional to the magnitude of interlayer coupling. Our detailed plan of presentation is given below.\\
%Most of the theories that have been put forward were based on the order-disorder mechanism dictating the ice-rules type physics in the low-temperature monoclinic phase. It may be noted that the order-disorder mechanism could be best explained by mapping the proton system to a quantum spin system comprising pseudo-spins. Here, the protons are represented by pseudo-spin variables corresponding to a suitable quantum spin Hamiltonian. As far as $\text{H}_2\text{SQ}$ is concerned, one of the prominent models that are currently gaining significance is based on the Ising lattice-gauge theory in (2+1)~\cite{chern2014gauge,kogut} dimensions with spins located on the edges of a unit square plaquette (see Fig.~\ref{lattice}(b)). Within the model, when only the quasi-2D nature was taken into account, the material was shown to host three different phases. As indicated, they are, 1) antiferroelectric (AFE), 2) quantum liquid-like, and 3) deconfined phase. \\
%\indent Apart from the fact that in a given layer, the dipoles (molecular polarization) are antiferroelectrically ordered, it is also known that the layers were stacked antiferroelectrically~\cite{hollander1977molecular}. Due to this fact, one may believe that the interlayer interaction to be of antiferromagnetic. However, in the present work, we show that even in cases that include disorder or ferromagnet, the material could support antiferromagnetic stacked configurations provided the interlayer interaction is weak compared to the intralayer interactions. Since the distortion of lattice at the critical point also could point out the non-uniformity of the interlayer coupling. We, therefore, invoke the non-uniform nature of interlayer coupling $J_3$ in our simulations and analyze the effect of it on the quasi-2D phases. We use the pseudo-spin formalism and employ the discrete version of the path-integral continuous-time quantum Monte Carlo technique to arrive at our results~\cite{nakamura2008efficient}. To distinguish three different phases,  we estimate various quantities such as the order parameters, $P$, $\rho$, and the imaginary-time correlation function, $c_\tau$, as a function of the transverse field, $K$. %It may be noted that $c_\tau$ measures the gauge-invariant correlation when the dipole-dipole interaction is turned off. We believe that the study of interlayer coupling is important as a sufficient interlayer interaction strength could lead to concerted proton tunneling between the interlayers dictating an alternative physics. \\
\begin{figure}[t!]
%\label{lattice}
\includegraphics[width=8.8cm,height=5cm]{Figures/3D_lattice.pdf}
\caption{(Color online) Schematic illustrating the detailed 3D lattice structure of $\text{H}_2\text{SQ}$. (a) Bird's eye view of $\text{H}_2\text{SQ}$ crystal lattice showing interlayers (labeled as $L-1,L,L+1$) coupled through the Ising exchange coupling $J_3$ (see (c)). (b) Schematic showing the lattice structure within a given quasi-2D layer. The small colored dots (blue) represent the protons
and while the compound $\text{C}_4\text{O}_4$ form a quasi-2D layered configuration with protons placed on the links of a dual lattice (green lines).  The local proton ordering which in the pseudospin language is $\mid \uparrow \rangle$, $\mid \downarrow \rangle$. (c) Cross-sectional view of the 3D $\text{H}_2\text{SQ}$ lattice structure. The indices $L-1, L, L+1$ correspond to $L-1^{\text{th}}, L^{\text{th}}, L+1^{\text{th}}$ layer respectively. Each square plaquette is occupied by exactly one square molecule where ice-rules are conserved by the gauge and Ising coupling constants, $J_0, J_1$.}     
\label{lattice}
\end{figure} 
\indent We begin by introducing the model in details in Sec. ~\ref{model} by briefly explaining the implementation details of the quantum Monte Carlo scheme.  Next in Sec.~\ref{quantities} we introduce and explain the different order parameters used to determine the presence of various phases such as global antiferroelectric phase and an intermediate quantum liquid phase. We also define imaginary-time correlation function which has been used to locate the  confined-deconfined phase transition.  In Sec. \ref{results}  we presents our results. First we discuss the effect of interlayer coupling on the spin-wave spectrum and spin wave velocity in both low and high-field limits. Next we present the resu lts of imaginary-time correlation function and obtain a qualitative understanding of the confinement-deconfinement transition as we increase the magnitude of external magnetic field. Following this, we present the antiferroelectric polarization order parameter, $P$ and ice-rule order parameter, $\rho$ (and corresponding susceptibility) versus temperature (and field) to draw the phase diagram in $T-K$ plane. Finally, we briefly present the dynamic structure factor obtained for two-dimensional layer where we argue its possible connection to the experiments.  In Sec.~\ref{discussion}, we discuss and briefly summarize our results .
%%%%%%%%%%%%%%%%%%%%%%%%%%%%%%%%%%%%%%%%%%%%%%%%%%%%%%%%%%%%%%%%%
                % Model and Method
%%%%%%%%%%%%%%%%%%%%%%%%%%%%%%%%%%%%%%%%%%%%%%%%%%%%%%%%%%%%%%%%%
\section{Model and method}
\label{model}
\subsection{Pseudo-spin Model}
Following the earlier model~\cite{ishizuka2011quantum,vijigiri2020dipole}, we consider a pseudo-spin model where the displacement of protons between the $O-O$ double-well potential is mapped on to the $z$-component ($\sigma^z_i$) of a spin-1/2 operator, $\sigma_i$, i.e, $\sigma^z_i=\pm 1$. Here $+1$ $(-1)$ refers to the proton being far from (near to) the center of the $O-O$ bond~\cite{blinc1960isotopic,horiuchi2008organic}. At zeroth order we consider a  four-spin interaction (with strength $J_0$)  as discussed in earlier studies~\cite{wesselinowa1995green} and has some intricate connection to the vertex-based models~\cite{stilck1981vertex}. The physical origin of the four-spin interaction term is primarily due to the fact that it yields the minimum energy configuration mediated by the $\pi$-electrons in $\text{C}_4\text{O}_4$ by the rearrangement of $\pi$-bonds. The model Hamiltonian also includes a next-to-next nearest neighbor (NNN) Ising interaction as well as a dipole-dipole interaction signified by the coupling strength $J_1$ and $J_2$ respectively. In addition to these terms~\cite{chern2014gauge}, in the present study we introduce an additional coupling between the two-dimensional layers represented by $J_3$ . The full Hamiltonian is thus given by,
\begin{eqnarray}
\label{Hamiltonian}
\mathcal{H} &=& -J_0\sum_{\square} \hat{A}_{\square} + J_1\sum_{\square}\hat{B}_{\square}  
 - J_2\sum_{\langle AB \rangle}\hat{\vec{P}}_{A}\cdot\hat{\vec{P}}_{B} \nonumber  \\
 && + J_3\sum_{L, i}\sigma_{L,i}^{z}\sigma_{L+1,i}^{z} - K\sum_{i}\sigma_{i}^{x} . %\\ 
 %\mathcal{H} &=& \mathcal{H}_{\text{2D}} + J_3\sum_{L, i}\sigma_{L,i}^{z}\sigma_{L+1,i}^{z}
\end{eqnarray} 
In the above $\hat{A}_{\square} = \sigma_{1}^{z}\sigma_{2}^{z}\sigma_{3}^{z}\sigma_{4}^{z}$, the indices $1$,$2$,$3$,$4$ correspond to the indices of four spins on the links (green lines in Fig.~\ref{lattice}) of a given unit-plaquette ($\square$) taken counterclockwise starting from the lower horizontal link($-$) (see Fig.~\ref{lattice}). The indices 1, 3, and 2, 4 are designated to be diagonally opposite to each other.  Similarly, $\hat{B}_{\square}=(\sigma_{1}^{z}\sigma_{3}^{z} + \sigma_{2}^{z}\sigma_{4}^{z})$. $\hat{P}_{A,B}^x=\pm\frac{1}{4}(\sigma^z_{1}+\sigma^z_{4}-\sigma^z_{2}-\sigma^z_{3})$, $\hat{P}_{A,B}^y=\pm\frac{1}{4}(\sigma^z_{1}+\sigma^z_{2}-\sigma^z_{3}-\sigma^z_{4})$ are the dipole-moment vectors for A, B sub-plaquettes as illustrated in Fig.~\ref{lattice}. As revealed in experiments, we note that the 2D layers are not stacked exactly one above the other but a finite shear displacement is observed alternatively among the adjacent layers (see Fig.~\ref{lattice}(a)-(c)). However, as the effective region  of interlayer interaction is associated with the H-bonds, we, therefore, consider the $J_3$ interaction to be of short-range (adjacent layers) composed of simple two-spin Ising interactions. We here consider the coupling $J_0$ as the largest among all the couplings  until stated explicitly, for the subsequent analysis carried throughout this work. We fix the value of $J_0$ equal to 1.  The four competing local interactions in Eq.~\eqref{Hamiltonian} except the Zeeman term (represented by $K$) are associated with different proton dynamics~\cite{vijigiri2018classical}. \\
\indent
Now we briefly describe the summary of previous work carried in the absence of the $J_3$ term. We request the interested readers to go through the Ref.~\onlinecite{vijigiri2020dipole,vijigiri2018classical} for more details. The four-body interaction term given by $J_0$ has eight degenerate spin configurations for a given plaquette. Among these eight, there are two configurations with all spins in 1/2 or all spins in -1/2 state and six configurations satisfying  \textit{ice-rule} where two spins are in 1/2 and rest of the other two spins in -1/2 state.  And in these six ice-rule states, there are four states that satisfy ice-rule and also have finite molecular polarization and the rest of the two states that do not have any finite polarization. It may be noted that the model described by the strong coupling term, $J_0$ and the external field $K_x$ is nothing but the quantum Ising gauge theory in (2+1) dimensions~\cite{chern2014gauge}. This theory is well-known to have a local $\mathcal{Z}_2$ symmetry where the degeneracy scales exactly equal to $2^{N_x+N_y+1}$~\cite{vijigiri2018classical}. The model shows a deconfinement to confinement transition as $K$ is increased from zero.  It is know that the addition of Ising type intramolecular interaction given by $J_1$ extends the deconfined phase further \cite{vijigiri2018classical}. The $J_2$ term in Eq.~\eqref{Hamiltonian} denotes a dipole-dipole interactions between two sub-plaquettes $A$ and $B$ as shown in Fig.~\ref{lattice}. This dipole-dipole interaction yields a global stripe ordering with four-fold degeneracy~\cite{vijigiri2018classical}. \\
\indent In an earlier study~\cite{vijigiri2020dipole}, it was found that in the presence of  dipole-dipole interaction, the Hamiltonian in Eq. \ref{Hamiltonian} with $J_3=0$ yields three different phases as the temperature (or field) is increased.  At low temperatures,   there exists an antiferroelectric phase up to a critical temperature. We note that in this global ordered antiferroelectric phase the ice-rules constraint is still satisfied but has finite dipole moment in order to minimize the $J_2$ interaction. As the temperature $T$ is further increased we find a quantum paraelectric phase (or the quantum liquid-like phase) where the long range order of antiferroelectric phase does not exist but the ice-rule configuration still survives locally maintaining a quantum coherence at short length scale. Finally as the temperature $T$, is increased further the thermal fluctuations prevail over the system and eventually the paraelectric phase is seen.  \\
\indent 
In the absence of $J_3$, the spin configurations  within the two-dimensional layers are independent of each other (layers). However, when $J_3$ is turned on, a certain correlation between the spin configurations of  nearest neighbour or adjacent layers may appear.  If one considers two isolated plaquettes which are connected vertically by the $J_3$ coupling, then one can naively understand that for either FM or AFM coupling the configurations are just aligned or anti-aligned to each other respectively. On the other hand if $J_3$ is disordered there may be more choices indicating increased entropy. However this isolated picture of  two plaquettes coupled vertically can not be generalized for two dimensional layers connected by $J_3$ bonds. %It is true that a given spin-configurations at a given plaquette can be repeated or anti-aligned depending on the nature of $J_3$, the dynamics gets consider quantum as well as thermal fluctuations. 
For example there are many degenerate configurations satisfying ice-rule configurations for a given two-dimensional layer. And when the layer goes from one degenerate configuration to another, this introduces certain length and time scale for the spins in adjacent layers to adjust. Thus this sets in some kind of rigidity against quantum and thermal fluctuations which could be the reason of high $T_c$ in experiments. In this respect present study seeks to quantitatively determine the dynamics of such coupled systems and how the quantum fluctuation and thermal fluctuation guides the systems to a new description of three dimensional squaric acid system.
%Intuitively, we understand that in general a) if the $J_3$ is FM, one would hope for a replica of configurations through all the layers. But for b) if the $J_3$ is AFM in nature, the layers could be trivially antiferroelectrically stacked. However, the effect on the intermediate state is not subtle, as it is purely a quantum state. And, for c) $J_3$ being disordered, one expects destruction of phases regardless of whether the state being globally ordered or a quantum superposition one. %\mandal{More on the physics due to this mechanism and a cartoon figure and then below paragraph will be completed. The cartoon figure will try to show constraints on ice-rules/dipole-dipole interaction due to AFM/FM/disordered strength of $J_3$.}{\color{blue}{This part is done: VIKAS,  updated figure, and text.}} \\
%%%%%%%%%%%%%%%%%%%%%%%%%%%%%%%%%%%%%%%%%%%%%%%
%\indent On the whole, to briefly outline the role of $J_3$, we note that the effect of it on the IR states within the 2D layers strictly depends on both the nature of the intercoupling ($J_3$) considered and also on the strength of it compared to the strengths of intralayer interactions ($J_{0,1,2}$). For example, in the AFM case, due to its frustrated tussle with $J_0$, we, therefore, believe that the entropy reduction near the intermediate liquid-like phase could be manifest (see Fig.~\ref{order_param}). Nevertheless, we proceed on to details of the simulation and the method used as briefed in the next section~\ref{method}.  \mandal{More on the role of $J_3$..what non-triviality it can bring. One thing is clear that when $J_3$ is much larger than anyone else that is for $J_3 \infty$ limit, the ice-rule should break down. Is that transition is of the first order or second order, is it possible to realize it in QMC. if yes very good otherwise, can we classically give an Adhoc estimate of that? One thing is sure, the possible ice-rule configuration in one layer now be coupled with the other layer..so the number of configurations will change...we need to estimate this. It means that entropy will be reduced. When there no coupling the entropy is higher than when the coupling is there. Can we define some parameters that can define this and in the abstract also we can write the results on this entropy change? External pressure can increase the value of $J_3$. can Paulson be of any help in doing a little bit of spin-wave calculation? It will be easy and doable within a month some primary results will be there.}  \color{blue}{Edited accordingly: VIKAS} \color{black}
%%%%%%%%%%%%%%%%%%%%%%%%%%%%%%%%%%%%%%%%%%%%%%%%%%%%%%%%%%%%%%%%%
                % Method
%%%%%%%%%%%%%%%%%%%%%%%%%%%%%%%%%%%%%%%%%%%%%%%%%%%%%%%%%%%%%%%%%
\subsection{Quantum Monte Carlo}
\label{method}
Here we briefly outline the method and detail implementation  of the Imaginary-time quantum Monte Carlo technique. We use the discretized version of the continuous time QMC algorithm developed by $Nakamura$~\cite{nakamura2008efficient}. Using the Suzuki-Trotter decomposition scheme~\cite{SUZUKI1993232}, the present three-dimensional quantum Hamiltonian given in Eq.~\eqref{Hamiltonian} is mapped on to an effective four-dimensional classical Hamiltonian where the additional fourth dimension is extended along the Imaginary-time (Trotter) axis $\tau$.  The effective classical action thus obtained is given by, 
\begin{eqnarray}
\label{action}
\mathcal{S} &=& \sum_{p}\bigg(-\frac{\beta J_0}{M}\sum_{\square} A_{\square,p} + \frac{\beta J_1}{M}\sum_{\square}B_{\square,p} \nonumber \\
&& - \frac{\beta J_2}{M}\sum_{\langle AB \rangle}\vec{P}_{A_p}\cdot\vec{P}_{B_p} + \frac{\beta J_3}{M}\sum_{\langle L_p\rangle}\sigma_{L,p}\sigma_{L+1,p} \nonumber  \\
&& - K^{\prime}\sum_{i}\sigma_{i,p}\sigma_{i,p+1} \bigg),
\end{eqnarray} 
%Studying the finite temperature property of an interacting spin system especially with a four-spin interacting Hamiltonian is not a  straightforward task. The analytical methods which rely on fermionization of the spins which map the original spin problem into interacting fermions face limitations of approximation methods and for such a spin system which contains a four-spin interaction usually results in six or eight fermion interacting models, thus where the analytical method became more intractable.
where at each Trotter index $p$ the operators initially of the form $\hat{A}_{\square}$ are mapped to $A_{\square,p}$. Note that the quantum spins are now replaced by their classical Ising variables, for instance,  $A_{\square,p}=\sigma_{1,p}\sigma_{2,p}\sigma_{3,p}\sigma_{4,p}$ and similarly the same is applied for all the remaining operators in the original Hamiltonian (Eq.~\eqref{Hamiltonian}). Here, $M$ is the cut-off length along Trotter axes and is fixed at the start of the simulation, while $\beta$ is the inverse temperature and $K^{\prime}=-\frac{1}{2}\ln \coth(\beta K/M)$ is the effective coupling (FM) along the Trotter dimension $\tau$.\\% The operators are now replaced by their classical variables of spin $\sigma_{i,p}$ with $i$ being the site-index and $p$ the imaginary-time.  \\
\indent We adapt the Swendsen-Wang (SW) \cite{swendsen1987nonuniversal} cluster update applied along the imaginary-time direction and interlayer axes. However, the case for $J_2=0$ is highly degenerate at smaller fields, and to avoid apparent spin-freezing and to speed up the relaxation process we employ the variant of the loop-flip update algorithm~\cite{Rahman}. Our variant of the algorithm involves loop that are formed at a particular imaginary-time slice randomly and extending them as SW clusters along the Trotter and interlayer axes. Once all such clusters are identified we update the configurations with their respective heat-bath probabilities. We note that for the case of $J_2\neq 0$ the loops are flipped with probability proportional to their Boltzmann weight since they are no more iso-energetic even though they form the subset of the restricted ice-rules. We consider system sizes up to $N=12\times12\times8$ block of spatial spins comprising the 3D lattice and the number of Trotter slices are fixed up to $M=1000$. About $\sim 10^5$ Monte Carlo steps(MCS) were used for equilibration and $5\times10^5$ were used for measurements. Binning analyses were done by dividing results into six bins where statistical errors are estimated. \\
\indent As we are interested in analyzing the system under both the quantum (fields) and thermal (temperature) fluctuations we, therefore, vary both the parameters taken along the contour that is parametrized by $\theta$, such that $\theta=\tan^{-1}(K\beta)$. For small inverse temperatures, the system is subjected to strong thermal fluctuations and vice-versa. We now proceed on to the next section~\ref{method} describing the details of physical quantities that will be estimated using the QMC method briefed here.
%%%%%%%%%%%%%%%%%%%%%%%%%%%%%%%%%%%%%%%%%%%%%%%%%%%%%%%%%%%%%%%%%
                % Physical Quantities
%%%%%%%%%%%%%%%%%%%%%%%%%%%%%%%%%%%%%%%%%%%%%%%%%%%%%%%%%%%%%%%%%
\section{Physical quantities}
\label{quantities}
%\begin{itemize}
%\item {\mandal{ Effect of disorder on correlation..a separate plot is expected.}}
%\item{\mandal{What happens that that intermediate quantum paraelectric liquid-like phase found in two dimensions.  Ok. I understand that it is now a zero temperature result. Can one comment on it..} \color{blue} This result is not a zero temperature one. As both the temperature and field are varied according to the $\theta=\arctan(K\beta)$}
%\end{itemize}
For convenience, we discuss the variations of parameters in two limiting cases  based on the ratios of $J_2$ and $J_3$. For the case $J_1\ggg J_2$, we measure the gauge-invariant dipole-dipole correlation function $c_\tau$~\cite{chern2014gauge} averaged over all molecules along the imaginary-time direction. The  correlation function $c_\tau$ characterizing the deconfined phase is given by,
\begin{eqnarray}
c_\tau=\frac{1}{N}\sum_{i}\langle P^x(i,0)P^x(i,\tau) \rangle,
\label{imcor}
\end{eqnarray}
where $\tau$ denotes the imaginary-time index, $N$ being the no. of molecules/plaquettes. The correlation function contains terms that are gauge-invariant and is used to characterize the deconfined phase. For small $J_2$ other parameters of the Wilson-loop operator form could also be used to distinguish the corresponding strong-coupling limit and weak-coupling limit (within the deconfined regime) where the correlation function follows perimeter and area-law respectively~\cite{kogut}. However, for $J_{2,3}=0$, the minimal Wilson-loop one can consider in spatial direction is the plaquette operator $\mathcal{P}=\sigma^z_1\sigma^z_2\sigma^z_3\sigma^z_4$ and the corresponding correlation would describe CDT.  Taking advantage of the gauge-fixing along the temporal axes to be one, the dipole-dipole correlation $c_\tau$ along the imaginary-time axis is now gauge-invariant and which is what we estimate in our simulation. Next, for finite $J_2$ we measure the order parameter $P$ along with the associated susceptibility given by $\chi_{zz}$  which characterizes the global antiferroelectric ordering developed by the dipole-dipole interaction term. 
In addition we also measure the off-diagonal expectation value of the magnetization along the transverse-field, $m_x$ along with the associated susceptibility $\chi_{xx}=\frac{\partial m_x}{\partial K}$,
\begin{eqnarray}
P &=&\frac{1}{N}[\mid S(0,\pi)\mid^2 + \mid S(\pi,0)\mid^2]^{1/2} \\
\frac{\chi_P}{N}&=&\beta\bigg[\langle P^2\rangle - \langle P^2\rangle\bigg]
\end{eqnarray}
where $S(\textbf{k})$ is the static spin-structure factor given by
\begin{eqnarray}
S(\textbf{k})&=&\frac{1}{N}\sum\limits_{i,j}^{N} S_{i}^z S_{j}^z\exp{(-\textbf{k}\cdot\textbf{r}_{ij})} 
\end{eqnarray}
here, $\textbf{r}_{ij}$ is the relative distance vector between $i^{\text{th}}$ and $j^{\text{th}}$ spin on the lattice. We also measure parameter $\rho$ which detects an ice-rule state locally and gives information about the amount of disordered dipoles configurations in general. This along with $P$ becomes crucial in understanding the local ordering of the protons. The relation for $\rho$ is given as, %(\mandal{To elaborate more on the $\rho,~\tilde{\rho}$ and why it becomes -1/3..it seems defining interlayer $\rho_{2d}$ in the absence of $j_3$ comparing it with the presence of $J_3$ will expose much physics or yields new insight. }..any other quantity that might include three dimensional ordering or changes ),
\begin{align}
    \tilde{\rho}_\square= 
\begin{dcases}
    1               & \text{if } \square \hspace{0.1cm}  \epsilon \hspace{0.1cm} \text{ice-rule}\\
    -1/3,              & \text{otherwise}
\end{dcases} 
&& \& \hspace{0.3cm}
\rho = \sum_{\square}\tilde{\rho}_\square,
\end{align}
though $\rho$ and $P$ would suffice in detecting any global ordering especially antiferroelectric. Finally, to locate the critical points accurately, we estimate the binder cumulant which is given by~\cite{binder1981finite},
\begin{eqnarray}
Q &= & \frac{1}{2}\Bigg( 1-\frac{\langle P^4
\rangle}{3\langle P^2 \rangle^2}\Bigg)
\end{eqnarray}

%%%%%%%%%%%%%%%%%%%%%%%%%%%%%%%%%%%%%%%%%%%%%%%%%%%%%%%%%%%%%%
%%%%%%%%%%%%%%%%%%%%%%%%%%%%%%%%%%%%%%%%%%%%%%%%%%%%%%%%%%%%%%%%% 
                % Results
%%%%%%%%%%%%%%%%%%%%%%%%%%%%%%%%%%%%%%%%%%%%%%%%%%%%%%%%%%%%%%%%%
\section{RESULTS}
\label{results}
\subsection{Discussions on classical ground state}
\indent \indent  We first discuss the effect of interlayer coupling semi-classically, particularly its effect on spin-wave spectrum and spin-wave velocity within the approximation by considering the terms that are quadratic in $a$'s. To do that fist we need to have an understanding about the classical ground states of model Hamiltonian given in Eq. \ref{Hamiltonian} around which spin-wave spectrum will be estimated to account for the quantum fluctuation around these classical spin configurations. Following Ref.~\cite{vijigiri2018classical}, we notice that the 3D-model Hamiltonian can also be rewritten in the following form: $H = \sum_i (h^z_i S_i^z + h^x_i S_i^x)$, where for a given spin component $S_i^\alpha$, $h^\alpha_i$ denotes the local-field component along $\alpha$ axis. The minimum energy configuration
of spins can then be obtained by aligning $S_i^\alpha$ to the negative $\alpha$ axis. We follow the same procedure that is given in Ref.~\cite{vijigiri2018classical}. We have performed numerical simulations over a lattice of dimension $16 \times 16 \times 16$ and inspected for adequate initial configurations. It is observed that in the presence of external field $K$,  all the spins would have a finite and constant value of $S_i^x$, which changes as a function
of $K$~\cite{vidal2008perturbative,vijigiri2018classical}. Therefore, the groundstate configurations
can be written as,
\begin{eqnarray}
\label{gs}
\vec{S}_i=S(\lambda_i \cos\theta \bf{e}_z + \sin\theta \bf{e}_x),
\end{eqnarray}
where $\lambda_i$ could be $\pm 1$ in tune with the ground state configurations of $H_0$ for $K=0$.  This  the configuration of $[{\lambda_i}]$ tries to take into account the other all the terms in the Hamiltonian except the Zeeman field term and a finite $\theta $ takes into account the Zeeman field induced term. The equilibrium configuration, that is, the value of $\theta_{\text{eq}}$ depends on $K$, $J_1$, $J_2$, and $J_3$. For $K=0$, we have $\theta=0$ but $\lambda$'s can still take either $\pm 1$. The $\theta$ plays the role of order parameter. From the mean-field ansatz given in Eq.~\eqref{gs}, the ground state energy of the system can be written as follows:
\begin{eqnarray}
\label{gsenergy}
E_{cl}&=&- \frac{1}{2}J_0S^4N\cos^4\theta - J_1S^2N\cos^2\theta - 2J_2S^2N\cos^2\theta \nonumber\\ 
&& - J_3 N S^2 \cos^2\theta - KSN \sin \theta .
\end{eqnarray}
\textcolor{red}{IN THE ABOVE EQUATION I THINK $|J_3|$ WILL APPEAR. PLEASE CHECK. BECAUSE WE CAN TAKE THE SPIN CONFIGURATIONS OF TWO NEAREST PLANE SUCH THAT THE $J_3$ BONDS HAS NO FRUSTRATION. THIS IS ALSO CONSISTENT WITH SPIN WAVE RESULTS.}
\textcolor{blue}{Now I am convinced that in the above equation $J_3$ should appear. Because when the two layers are FM/AFM way connected Eq. 9, the ansatz will have a $\pm$ signs in front of $\lambda_i$.}
\indent Minimizing $E_{cl}$ with respect to $\theta$, we obtain $\theta_C$, which minimizes the ground-state energy $E_{cl}(\theta_C)$. This ground-state energy has been compared with the $E_x=-KN$, which denotes the energy corresponding to the state where all spins are aligned along $x$-direction. For a given $J_0,J_1,J_2$, there exists a $K_c$ such that if $K\le K_c$ then $E_{cl}<E_x(\theta_C)$ with $\theta_C\le \pi/2$. One can numerically solve for $\theta_C$ and obtain the classical phase diagram~\cite{vijigiri2018classical}. \textcolor{red} {If the Hamiltonian does contain only $J_0$ and $K$, the $\theta_c$ denotes a confinement-deconfinement transition. For in this limit for $K < K_C$, the distribution of $[\lambda_i]$ minimizes four body interaction given by $J_0$. The states includes predominantly ice-rules satisfying state. In two dimension it has been found that inclusion of $J_1$ and $J_2$ increases the stability of the   deconfined phase governed by $J_0$ terms. Though inclusion of $J_1$ and $J_2$  puts certain restriction on ice-rule states, deconfined phase survives and become more stable indicating a larger value of $\theta_c$. From the mean-field decomposed classical energy expression given in Eq. \ref{gsenergy}, we find the inclusion of an antiferromagnetic (positive) $J_3$ further stabilises the deconfined phase as it adds up  directly to
the energy contribution by $J_0, J_1, J_2$.  On the other hand if $J_3$ is taken ferromagnetic then it acts apposite to the energy contribution  by $J_0, J_1, J_2$. Thus it destabilizes the deconfined phase and reduces the value of $\theta_c$.}  \\
\indent  The classical ground-state has finite degeneracy in the presence of $J_0, J_1$, and $J_2$ which remains unchanged after inclusion of interlayer coupling $J_3$. For finite interlayer coupling, that is, $J_3>0$ $J_1=J_2=0$, the degeneracy is still $2^{N_xN_y+1}$  and for $J_2=0$, the degeneracy is $2^{N_x+N_y}$. For both $J_1$ and $J_2$ nonzero, the degeneracy is reduced to four as in the case of 2D case~\cite{vijigiri2018classical}. For large $K>Kc$, all the spins get aligned along the $x$-axis corresponding to $\theta_c=\pi/2$. The degeneracy of 3-dimensional system follows directly the degeneracy of a any  two dimensional plane for ferromagnetic and anti-ferromagnetic interlayer coupling. This is seen in the simulations where the specific heat shows an anomalous peak whose height is proportional to this degeneracy. \textcolor{red}{The degeneracy of three dimensional system is not straightforward to calculate for a disordered configurations of interlayer coupling.  In the disorder case, the spin-configuration of two nearest-neighbour two dimensional plane are frustrated. However it could very much happen that while the spin configurations of any two-dimensional planes are frustrated, there are many degenerate configurations which all are frustrated but have the same energy. Interestingly it may also happen within this degenerate manifold of frustrated ground state configuration a certain two dimensional plane is not frustrated. Thus the exact determination of effect of disordered interlayer coupling on the degeneracy is very subtle. However later we will observe that the specific heat shows a reduced anomalous  peak at lower temperature which asserts the fact that some residual degeneracy survives.} 
\subsection{Linear spin-wave theory}
\indent \indent In the previous section we have found that the classical ground-state spin configuration  obtained is  dependent on $\theta$ implying that we can define a local axis represented by $x^{\prime} (z^{\prime})$. This $(x^{\prime}, z^{\prime})$ is obtained from $(x,z)$ by a rotation by angle $\theta$ around $y$-axis. The spin operators in prime co-ordinate is defined by $\bf{S}^{x^\prime}_r = s - \bf{a}^\dagger_r\bf{a}_r,  \hspace{2mm} \bf{S}^{z^\prime}_r = \sqrt{s/2}(\bf{a}^\dagger_r + \bf{a}_r) $ within large $\bf{S}$ approximation \cite{Ref}. \textcolor{red}{Here $ \bf{a}^{\dagger}$ denotes the creation of magnon operators which are defined along the  directions of alignment of local $z^{\prime}$-axis.}  To make use of this definition we have to rotate the local axis to the global co-ordinate systems, which is given as,
\begin{eqnarray}
\label{rotation}
\bf{S}^x_r &=& \bf{S}^{x^\prime}_r \cos\lambda_i\theta - \bf{S}^{z^\prime}_r \sin\lambda_i\theta, \\
\label{rotation1}
\bf{S}^z_r &=& \bf{S}^{x^\prime}_r \sin\lambda_i\theta + \bf{S}^{z^\prime}_r \cos\lambda_i\theta. 
\end{eqnarray}
\begin{figure*}
\includegraphics[width=5cm,height=3.5cm,valign=t]{Figures/sw_1.pdf}
\includegraphics[width=4.2cm,height=3.5cm,valign=t]{Figures/sw_2.pdf}
\includegraphics[width=4.2cm,height=3.5cm,valign=t]{Figures/sw_3.pdf}
\includegraphics[width=4.2cm,height=3.5cm,valign=t]{Figures/sw_4.pdf}
\caption{\label{square} Plot showing the dispersion in the high-field case where quadratic behavior slowly converges to a linear behavior at the
second-order critical line given by $K = 2sJ_1 + \frac{21}{8}sJ_2+2s|J_3|$. The spectrum is plotted for various values of $J_2$, the rest of the parameters used in plot (a) are: 
$J_3 = 0$, $K = 0.5$, $J_1 = 0.5$,  $J_0 = 1.0$. Similarly for plot (b),  $J_3 = 0.2$, $K = 0.5$, $J_1 = 0.5$, $J_0 = 1.0$. And for plot (c), $J_3 = 0.8$, $K = 0.5$, $J_1 = 0.5$, , $J_0 = 1.0$. The various symmetry point used in the above figure is as follows: $\Gamma$ =
$(0, 0, 0)$, $R$ = $(\pi, \pi, \pi)$, $X = (0, \pi, 0)$, and $M = (\pi, \pi, 0)$. \textcolor{blue}{There are some problems with the labelling. Label (c) is missing. I think the plot which is labelled as (b) should be (d). Which is labelled as (d) should be (c) and which is labelled as (a) in the lower left panel is (b). By this we can only attribute the abrupt or completely different spin-wave spectrum shown in the right upper panel. This is for maximum $J_3$ and this shows minima around (R) point. also from the input file sw$_1$.pdf has been used twice. Also in the y-axis $\Delta_k$ has been used. Is it same as $E_k$ or not?}}
\label{highfield}
\end{figure*}

%\begin{eqnarray}
%\label{hptransform}
%\bf{S}^{x^\prime}_r = s - \bf{a}^\dagger_r\bf{a}_r,  \hspace{2mm} \bf{S}^{z^\prime}_r = %\sqrt{s/2}(\bf{a}^\dagger_r + \bf{a}_r).
%\end{eqnarray}
\indent Once  we substitute the above definitions in to the Hamiltonian we obtain the so called spin-wave Hamiltonian. We note that spin-wave Hamiltonian contain quadratic term along with quatric and cubic term. In the following we only retain the terms upto quadratic to make further progress.\\

\subsubsection{\textcolor{blue}{Discussion in the small field limit}}
\textcolor{blue}{Previously in the two dimensional case we have found that in the small field limit, the spin wave spectrum does not cause any order from disorder transition. The macroscopically degenerate ground state manifold have identical ground state energy under quantum fluctuation in quadratic approximation. This happens due to the presence of macroscopic conserved quantities discussed previously \cite{vijigiri2018classical}Now some comments about three dimensional case will be enligtening and is a very important point. What we all need is to tame few $N_x \times N_y \times N_z$ systems and calculate the ground state energy and check whether there are some order from disorder transition. I guesst that there will be some transition and this will be one of the main results. }
\subsubsection{High-field limit, $K \gg J_0,J_1,J_2$ \& $\theta_0=\pi/2$}
\indent \indent The groundstate in this limit is a trivial one with all the spins pointing towards the $x$-axis. To obtain the magnon-spectrum and to check the stability of the classical groundstates, the Holstein-Primakoff transformation Eq.~\eqref{hptransform} is substituted into the classical version of the  Hamiltonian~\ref{Hamiltonian} with the groundstate Eq.~\eqref{gs} now being explicitly embedded into it. Since in the high-field case we invoke the translation invariance of the spin-configurations, we then can Fourier transform the $a_i$ variables to $a_k$ as $a_i=\frac{1}{N}\sum_{\vec{k}}a_{\vec{k}}\exp({-i\vec{k}\cdot\vec{r}})$. Substituting the transformed boson operators into the above equation yields a Hamiltonian of various orders in $a$. Within quadratic approximation we neglect the higher order terms beyond second order, the Hamiltonian then is given by, 
\begin{eqnarray}
\mathcal{H}&=&N\sum\limits_{k}\bigg[\xi_ka_ka^\dagger_k + \frac{\gamma_k}{2}(a_ka_{-k} + a^\dagger_k a^\dagger_k)\bigg] +  C_0 + \mathcal{O}(a^3)\nonumber \\
&&+ \mathcal{O}(a^4),
\end{eqnarray} 
where $C_0=KSN$ and is the classical energy upon which the terms within the bracket has the information of the first dispersion in $k$-space, given by $\xi_k$. The $\xi_k$ and $\gamma_k$ is given below as,
\begin{eqnarray}
\xi_k&=&\gamma_k + K -\frac{SJ_2}{8},\\
\label{pk}
\quad \gamma_k &=&\frac{-s}{4}\big[J_2(2 p_k^2 -1) - 4(J_1+J_2)p_k\nonumber \\
&&+ 4|J_3|\cos(k_z)\big],
\end{eqnarray} 
where $p_k=\cos(k_x+k_y)\cos(k_x-k_y)$. Now, the Hamiltonian is easily diagonalizable and upon doing that we obtain the eigen energies of the magnon spectra as given by,
\begin{eqnarray}
\label{dispersion}
E_k&=& \sqrt{\xi^2_k - \gamma_k^2}.
\end{eqnarray}
\indent In the Fig.~\ref{highfield} we show the spectrum plotted for parameters $J_1=0.2, K=1.0$ and varying $J_2$ from $0.1$ to $0.5$ \textcolor{red}{This is not consistent with what is said in Fig. \ref{square}}. To qualitatively decipher the linearity seen in the dispersion curves obtained for $J_2=8/21$ we find the spectrum by extracting the low-energy behavior around the minima, that is, $X$ (or $R$) high-symmetry point. We consider the following approximation $(k_x=-\pi/2 + \delta_x, k_y=\pi/2 + \delta_y, k_z=\pi + \delta_z)$ around $R$. Substituting these in Eq.~\eqref{dispersion} and expanding the terms $\gamma_k, p_k$, we get,
\begin{eqnarray}
\label{linearspectrum}
E_{\vec{\delta}}=\tilde{K}_1^{1/2}\sqrt{\tilde{K}_2 + 4(J_1 + 2J_2+|J_3|)|\vec{\delta}|^2},
\end{eqnarray}
here, $\tilde{K}_1=K-\frac{sJ_2}{8}$, $\tilde{K}_2=(K-2sJ_1-\frac{21}{8}sJ_2+2s|J_3|)$. From the Eq.~\eqref{linearspectrum} we can see that for all the values of $J_{1,2}, K$ the spectrum remains gapped with low-energy quadratic behavior in the dispersion curves except at the second-order transition point as obtained by $K_c=2sJ_1 + 21/8J_2+2s|J_3|$ \textcolor{blue}{This is also consistent with the results of classical analysis that $k_c$ increases with $J_3$, the deconfined phase gets more stable. But this $K_c$ is unusually high}. The spectrum at this point becomes linear and gapless, indicating a possible signature of with the underlying features of the groundstate for higher $J_2$, which is the antiferroelectric case. This also shows that the transition for smaller values of $J_2$ the system behaves quite differently from the one with relatively higher values of $J_2$. This can be attributed to the fact that the gapped spectrum is protecting the degeneracy in the groundstate manifold in the absence of dipole-dipole interaction $J_2$, with the excitations being gapped and discrete. \textcolor{red}{When $J_2$ is increased and approaches the antiferroelectric ordered state, the spin-wave spectrum becomes linear and gapless as expected.BUT FROM THE FIGURE 2. INCREASE OF $J_2$ SEEMS TO MAKE THE SPECTRUM MORE OR MORE FLATTER. THIS IS BECAUSE OF $J_3$ OR NOT? IT SEEMS THERE IS SOME ESSENTIAL DIFFERENCES BETWEEN TWO DIMENSION AND THREE DIMENSONS AS FAR AS THE ACTION OF $J_2$ IS CONSIDERED. IN TWO DIMENSION INCREASE OF $J_2$ MAKES THE SPECTRUM LINEAR BUT IN THREE DIMENSIONS IT IS BECOMING MORE GAPPED OUT DUE TO THE PRESENCE OF $J_3$.} And as one increases the value of $J_2$ further we see that the $\tilde{K}_2$ changes to negative in sign where the spectrum becomes imaginary~\eqref{linearspectrum}. The same can be deciphered from the Fig.~\ref{highfield} shown in dotted lines near the $X(R)$ high-symmetry points where there are no states available with the gap $\Delta(k)$ being undefined. It is to be noted that for sufficiently large $J_2$ one needs to perform the SW analysis on a different groundstate. Importantly, from Eq.~\ref{linearspectrum}, we can see that the spin-wave velocity is independent of the nature of the coupling. Since,
\begin{eqnarray}
\frac{\partial(E_{\vec{\delta}})}{\partial{\vec{\delta}}}\propto |J_3\vec{\delta}|\tilde{K}_1^{1/2}\sqrt{\tilde{K}_2 + 4(J_1 + 2J_2+|J_3|)|\vec{\delta}|^2}.
\end{eqnarray}
\indent\textcolor{red}{The indistinguishability of the spin-wave velocity can be understood as follows: Let us consider a single layer, in a ground state of $J_0$, where each plaquette can take up to 8 different configurations with the same energy (four of them are shown in Fig.~\ref{rho_ice_rules}(a)). Now, if we pick any one of the eight states we see there exists one high-energy (from the rest of the seven states) state that appears at finite temperatures or fields. And, six of the low-energy choice where the system can exist at low-temperatures (from the remaining six other choices). Similarly, we do the same analysis for the AFM case and we see the same number of choices are available even for that case. That is even for AFM coupling there exists six low-energy choices and one high-energy choice the system can exist. Hence, we note that this choice of states is independent of the nature of the coupling, i.e, it does not distinguish whether the coupling is AFM or FM. This is reflected in the spin-wave spectrum and thus we obtain identical spin-wave velocity.THE ABOVE ARGUMENT DOES NOT LOOKS CONVINCING. IT HAS TO DO WITH SOME SYMMETRY SUCH THAT IN EQ. \ref{pk} ONLY MOD $J_3$ APPEARS. THIS IS SOMETHING TO $Z_2$ SYMMETRY. now i doubt whether the sign of $J_3$ as taken in Eq. \ref{gsenergy} is taken correctly.}\\
\indent
\textcolor{blue}{COMMENTS:  The results on classical ground states by estimating $K_c$ for various parameters and comparing with the $K_c$ obtained from spin-wave analysis and further comparing with $K_c$ obtained from the monte-carlo simulations of correlation function(as given in Eq. \ref{imag-time-correlation}) will give a very nice way to understand the effect of quantum fluctuation and role of FM, AFM or disordered coupling. This is in our hand. All we need to do is take a consistent choice of parameter values in all the three analysis. This will be taken as an important motivation to study different methods, in the abstract we can also write that we determine the $K_c$.}
%%%%%%%%%%%%%%%%%%%%%%%%%%%%%%%%%%%%%%%%%%
\subsection{Imaginary-time quantum Monte Carlo}
\indent \indent  First, we consider the confinement-deconfinement transition (CDT) where $J_2=0$. We vary the external field for various interlayer couplings, and calculate the imaginary-time correlation $c_\tau$ as a function of $\tau$, as shown in Fig.~\ref{imag-time-correlation}. Similarly, in the next section, we provide the results for the order parameter $P,\rho$ and specific heat $C$ for $J_2\neq 0$ where the material is believed to host the AFM phase.  
%%%%%%%%%%%%%%%%%%%%%%%%%%%%%%%%%%%%%%%%%%%%%%%%%%%%%%%%%%%%%%%%%
                % CDT transition
%%%%%%%%%%%%%%%%%%%%%%%%%%%%%%%%%%%%%%%%%%%%%%%%%%%%%%%%%%%%%%%%%
\begin{figure}[t!]
\centering
\hspace{-0.1cm}
%\includegraphics[width=.5\textwidth]{order_param.pdf}
\includegraphics[width=8.0cm,height=12.0cm]{Figures/corr_plot_.pdf}
\caption{Plots showing the variation of the correlation function $c_{\tau}$ along the imaginary-time axis as estimated for different values of field strength. $J_0=J_1=1.0$ is used in all the figures here. (a) Corresponds to the case where the strength of the interlayer coupling $J_3=0.1$ with a critical value of $K_c=1.63$.  Similarly, (b) for $J_3=0.3$. We see that the correlation function decays according to the power-law for low-field values, that is, for $K=0.6,0.7$ ($K<K_c=0.94$) and then an exponential decay for $K=1.0,1.1,1.2$ ($K>K_c$). \textcolor{blue}{here we see that as we increase the value of $J_3$, value of $K_c$ decreases. This means the deconfined phase is destabilized as we increase the value of $J_3$, this is against the classical analysis which seems to suggest that $K_c$ increases as we increase $J_3$. However the results could be all correct because in the quantum simulation proper account of three dimensional structure is considered and the fluctuation of spins at nearest neighbour plane helps to confine the protons easily, more to thin. Physically we need to think how the increase of $J_3$ helps in confining, or help assisting $K$ in confining the protons.  }} %\mandal{What is the Kc..is there any numerical values for a given set of parameter. Also a natural question arises to take the $J_3$ limit zero. Naturally one would be interested to know whether the interlayer coupling increases or decrease the correlation value. } \color{blue}{Edited accordingly: VIKAS}}\label{order_param}
\label{imag-time-correlation}
\end{figure}
\subsubsection{Confinement-deconfinement transition}
\indent \indent In Fig.~\ref{imag-time-correlation}(a-b), we plot $\log c_\tau$ versus $\tau$ for various values of field strengths. The plots are shown for $J_3$ where the coupling is taken as disordered type. We see that when the field strength is below a critical field of $K_c=1.63$ (Fig.~\ref{imag-time-correlation}), the $\log c_\tau$ curve starts deviating from the power law decay (a behavior corresponding to a deconfined phase~\cite{chern2014gauge}). The functional form we fit is given by $c_\tau=r^{-n}\exp{r/r_0}$ (here, $n\sim 0.61$), that is, a behavior intermediate between a power law and exponential decay. We distinguish both the phases by the fact that in the paraelectric phase the behavior one expects is purely exponential decay (cases where $K=2.0,2.1,2.2$ in Fig.~\ref{imag-time-correlation}(a)). Similarly, in Fig.~\ref{imag-time-correlation}(b) where the strength of $J_3$ is now higher and is equal to 0.3, we see the function $c_\tau$ shows a lesser amount of long-range correlation for $K=0.6,0.7$ ($K<K_c$) decaying slowly away from the power-law $\sim$ 1/$r^{\eta-d/2+\nu}$ compared to Fig.~\ref{imag-time-correlation}(a). This may be due to the reason that the quantum fluctuations in higher dimensions are much more pronounced leading to such a behavior.  However, intuitively, when the field strength is further increased we see that the $c_\tau$ deviates more and more away from the power-law (see the blue dot curve in Fig.~\ref{imag-time-correlation}) as it decays according to the exponential law corresponding to a confined phase. The critical point for $J_3=0.1$ is $K_c=1.63$ and similarly for $J_3=0.3$ is $K_c=0.94$.  \\
\begin{figure*}[t!]
\centering
\includegraphics[width=1.0\textwidth]{Figures/all_T_0.5_0.05_0.03_16_1_8.pdf}
%\includegraphics[width=0.3\textwidth]{Figures/AFM2.pdf}
%\includegraphics[width=0.3\textwidth]{Figures/AFM3.pdf}
%\includegraphics[width=0.33\textwidth]{Figures/FM1.pdf}
%\includegraphics[width=0.3\textwidth]{Figures/FM2.pdf}
%\includegraphics[width=0.31\textwidth]{Figures/FM3.pdf}
%\includegraphics[width=0.32\textwidth]{Figures/DIS1.pdf}
%\includegraphics[width=0.3\textwidth]{Figures/DIS2.pdf}
%\includegraphics[width=0.31\textwidth]{Figures/DIS3.pdf}
\caption{The variation of the order parameter $P$ ((a), (e), (i)), the ice-rule detecting parameter $\rho$ ((b), (f), (j)) and their corresponding susceptibilities $\chi_P$ ((a), (e), (i)) and and $\chi_\rho$ ((b), (f), (j)) as a function of temperature (and field) are shown for different lattice dimensions. Similarly, the specific heat over temperature, $C/T$ is shown in (c), (g), (k) panels. Finally, the uniform magnetic susceptibility is shown in (d), (h), (l) correspondingly. The variation of all the parameters are estimated for three cases as labeled along the vertical axis, that is, AFM (antiferromagnet), FM (ferromagnet), Disorder. All the plots include the same, $\theta$, that is, $\theta=\pi/6$. The values used in each of the cases are: $J_1=0.5$, $J_2=0.04$, $J_3=0.4$ with $8$, $10$, $12$ lattice dimensions used along the interlayer axis, while we fix $12\times 12$ as the intralayer dimensions.}\label{order_param}
\label{order_param}
\end{figure*}
\begin{figure}[t!]
\centering
%\hspace{-1.5cm}
%\includegraphics[width=0.40\textwidth]{Figures/ice_param.pdf}
\includegraphics[width=0.48\textwidth]{Figures/AFM_TF_0.5_0.03_0_8_16.pdf}
%\includegraphics[width=9.5cm,height=15.0cm,left]{corr_plot_.pdf}
\caption{(Color Online) Variation of the order parameter $P$ and the ice-rule parameter $\rho$ versus field (temperature) are shown for different cases of $J_3$. The interlayer coupling values used are $J_3=0.1,0.4,0.8$ in the increasing trend of the color contrast of respective parameters (e.g: orange color corresponds to $P$ and the increase in $J_3$ can be seen from the least to the highest contrast). a) Corresponds to the variation with AFM, b) FM and c) disordered interlayer couplings}
\label{ice_param}
\end{figure}
\subsubsection{Antiferroelectric phase transition}
%\mandal{Physical reason for choosing the parameter value for $J_3$ and $D$ (We should represent $D$ by $\delta$). It seems sometimes $D$ is greater than $J_3$. Ideally, it should be 10 or 20 percent of $J_3$.}
\indent \indent \textit{Antiferromagnet coupling:} In Fig.~\ref{order_param}(a-d) order parameters $P, \rho$ and corresponding susceptibilities $\chi_P, \chi_\rho, C, \chi_s, Q_P$ are plotted against $K$. Note that not only the temperature is varied but also the field as dictated by the relation $K=T\tan\theta$. So, when a certain temperature value is mentioned, it is to be understood that the corresponding temperature value is $K=T\tan\theta$. Here for Fig.~\ref{order_param} $\theta$ is adjusted to $\pi/6$. We see that the $P$ has a maximum value of 1 till the temperature of 0.3, indicating an antiferroelectric order. As the temperature is increased further, the $P$ shows a plateau behavior for intermediate temperatures (around $K$ $\sim 0.2$ to $0.33$) before vanishing to a conventional paraelectric state ($K>1$). The plateau behavior can be identified as an intermediate state with strong quantum fluctuations~\cite{vijigiri2020dipole,ishizuka2011quantum}. This is because the order parameter $P$ gets vanished around these temperatures while the ice-rule parameter $\rho$ continues to show a constant value of $\rho=1$ till the temperatures of $K=1.1$. So the region of temperatures where the $P$ vanishes to the temperature where the value of $\rho$ starts decreasing from 1 should correspond to a state which satisfies ice-rules. Also, these states should carry a zero net dipole-moment (molecular polarization). This is only possible when each plaquette has a finite dipole moment (molecular polarization) but is randomly oriented with respect to the neighboring plaquettes. Similarly, the values for susceptibility, $\chi_P$ also corroborate the same as a peak around $~$0.25 seems to suggest a possible second-order transition at $K_c=1.70(15), 1.85(27), 2.04(35)$ for $J_3=0.2, 0.3, 0.4$ respectively. Note that the scaling of parameters $P,\rho$ shows a uniform behavior as we vary the linear size of the system from $L_x=8$ to $L_x=12$. For $L_x=8$, the plateau is steeper than that for $L_x=$12. This is intuitive, as the higher sizes are understood to have fewer fluctuations with stable configurations. \\
\indent  Specific heat, $C$ (see Fig.~\ref{order_param}(c)), also shows a small anomalous peak similar to the behavior observed with no interlayer coupling. The susceptibility obtained due to $P$ is shown in Fig.~\ref{order_param}(a)
and (b) by light red colored points which shows a jump at the transition from ferroelectric phase to quantum liquid-like states. This suggests that ferroelectricity is almost destroyed at this transition. However the susceptibility corresponding to $\rho$, that is. $\chi_\rho$ is initially at lower temperatures it is almost zero, which gradually increases until the temperature reaches near the transition from quantum liquid-like states to the paraelectric states. At this transition the $\chi_\rho$ jumps at a higher value and remain almost constant up to a certain temperature which we call $T_{\chi_\rho}$ and after this, $\chi_\rho$ decreases monotonically. The specific heat at a very large temperature shows monotonically decreasing behavior characteristic to the usual paraelectric phase but at low temperature, it shows two peaks of different magnitude as seen in Fig.~\ref{order_param}(c). The largest peak appears at the transition of quantum liquid-like state and we denote this temperature by $T^\star_{C/T}$
. However, the sharp nature of the peak indicates a possible order-disorder phenomenon where the degeneracy seems to be uplifted to some extent. Below this
temperature specific heat shows another small peak at where the $P$ starts to decrease from the initial constant value for small $T$.   This may be
attributed to the fact that the increase of $K$ results in more quantum fluctuations. \\
\indent  In Fig.~\ref{order_param}(d), we plot the results for uniform susceptibility averaged over a single layer show an antiferromagnetic-like order along the orthogonal unit axes spanning the lattice. This shows that as the system size increases the peak height reduces and only vanishes in the thermodynamic limit.     \\
\indent \textit{Ferromagnet coupling:} Similar to the AFM case, the results are shown accordingly in Fig.~\ref{order_param}(e-h). Here also the order parameter $P$ shows a value close to 1 at low temperatures (and temperatures) indicating an antiferroelectric phase. But unlike the AFM case, as we increase the temperatures further the intermediate liquid-like state seems to extend to much larger temperatures compared to the AFM case, indeed the intermediate-liquid-like extends from values of $~0.3$ to $~1.2$. This can be understood that because of FM type coupling, one expects a lesser amount of quantum fluctuations (compared to AFM and disordered case) across the interlayers. This trend is followed similarly by susceptibility since it shows a peak around the same value of $K$ around $~0.25$. This is because we have used the same values of interlayer strength with just opposite signs and thus the peak around the same temperatures. \\
\indent Specific heat curves plotted in Fig.~\ref{order_param}(g) show that cross-over temperatures have slightly shifted to higher temperature values compared to the AFM case (see Fig.~\ref{order_param}(c)). This is in favor of the finding that the intermediate state extends to much larger temperatures in the curves of order parameter $P$. The common aspect in AFM and FM case is about the peak at low temperatures ($\sim 0.3$).     
The peak values are the same in both cases, unlike the peak appearing at higher temperatures. Which for the FM case is sharper than the AFM case. This may seem to suggest that a possible order-disorder phenomenon taking place driven by thermal energy. \\
\indent \textit{Disordered coupling:} In this case, we see that $P$ shows a behavior that is very close to the AFM and unlike the FM case. The only difference that we see is in the specific heat curves. We observe that the anomalous peak almost vanishes and also the crossover peak at later temperatures gets broader than the rest of the two cases. This strongly suggests that the disorder completely lifts the degeneracy and there is no re-entrance of the intermediate-liquid-like as such. The values of the specific heat curves are intermediate to that of AFM and FM. In all the cases, the variation of $\chi_s$ is antiferromagnetic. Though for the disorder the antiferromagnetic nature is slightly perturbed (see Fig.~\ref{order_param}(i-l)).\\
\subsubsection{Variation with the interlayer interaction strength}
\indent \indent \textit{Antiferromagnet coupling:} We also plot the behavior of $P$ and  $\rho$ for various values of $J_3$. In Fig.~\ref{ice_param}(a), we have shown the variation of $\chi_P$ and $\chi_\rho$ for $J_3$=($0.1,0.4,0.8$) in orange and navy blue colored points respectively. For large values of $J_3=0.8$, $P$ and $\rho$ have sharper peaks and decrease more rapidly compared to the case of small values of $J_3$. This could be due to the lowered stability of the intermediate liquid-like state as the coupling strength $J_3$ is increased. Also, though the plateau is diminishing, the value of $P$ at low temperatures remains the same for all the cases of coupling. The AFE phase under the AFM type coupling is stable while for the intermediate liquid-like state this is not the case. \\
\begin{figure*}[t!]
\centering
%\hspace{-1.5cm}
\includegraphics[width=1.0\textwidth]{Figures/AFM_phase_diagram.png}
\includegraphics[width=1.0\textwidth]{Figures/FM_phase_diagram.png}
%\includegraphics[width=9.5cm,height=15.0cm,left]{corr_plot_.pdf}
\caption{Phase diagram in $T_K$ plane for both the cases, that is, AFM (Fig.~(a)-(c)) and similarly for FM (Fig.~(d)-(f)). In each row starting from the left (Fig.~(a)), the cases correspond to $J_3=0.1,0.3,0.6$ of interlayer coupling strength. The labels \textbf{I}, \textbf{II}, \textbf{III} indicate the AFE phase, Intermediate liquid-like phase and conventional paraelectric phase respectively.}
\label{phase_diagram}
\end{figure*}
\indent \textit{Ferromagnet coupling:} In the Fig.~\ref{ice_param}(b), clearly the plateau region in order parameter $P$ extends to much larger temperatures as the coupling is increased. This is in contrast to the AFM case. When the value of $J_3$ is equal to 0.1, the intermediate liquid-like state extends from $T=0.42$ to $T=0.94$. When $J_3= 0.4$, it extends from $T=0.45$ to $T=1.4$. And for $J_3=0.8$, it is much higher. The reason could be because of the lesser quantum fluctuations in the FM case. Nevertheless, the order parameter $P$ still shows a value of 1 for all the values of coupling strength, $J_3$. This is similar to the AFM case.\\
\indent \textit{Disordered coupling:} In the Fig.~\ref{ice_param}(c), when the value of $J_3$ is small, i.e, 0.1, the variation of order parameter resembles slightly that of AFM and FM case. The plateau region extends from $T=0.25$ to $T=0.75$. But when $J_3$ is increased to $0.4$, we see that the intermediate liquid-like state only extends from $T=0.25$ to $0.6$. Similarly, when $J_3=0.8$, comparable to the energies of the intraplane interactions, the order parameter $P$ has a lower value at lower temperatures. Also, there is no plateau behavior, instead, the abrupt vanishing of $P$ is seen. This could be understood because of more quantum fluctuations induced by disorder type. However, for lower values of interlayer coupling, even the disordered type coupling hosts both an AFE and intermediate liquid-like state. The variation of ice-rule parameter $\rho$ is used to show that there exists an intermediate state with finite-molecular polarization since the value of $\rho$ close to $~$1 indicates that the entire lattice is in an ice-rule state (apart from statistical fluctuations) and any deviation from it indicates a non-ice-rule state. \\
\begin{figure*}[t!]
\centering
\hspace{-0.5cm}
\includegraphics[width=18.0cm,height=4.9cm]{Figures/density_plot.pdf}
\caption{The dynamic structure factor, $\mathcal{S}^{xx}(k,\omega)$ within the realm of Real-space linear spin-wave theory (RS-LSWT) calculated for weak interlayer coupling limit. In all the plots, we fix the value of $J_0=1$. a) For the parameters $J_1=1.0$, $J_2=0$, $K_x=0.3$. b) For $J_1=1.0$, $J_2=0.019$, $K_x=0.3$. Similarly, c) For $J_1=1.0$, $J_2=0.3$, $K_x=0.2$. Here, $\Gamma\equiv(0,0)$, $M\equiv(\pi,\pi),X\equiv(0,\pi)$}
\label{density_plot}
\end{figure*}
\subsubsection{Phase diagram}
\indent \indent \textit{Antiferromagnet coupling:} In Fig.~\ref{phase_diagram} (top row), we plot the critical points in the $T-K$ plane as obtained from the the Binder cumulant (see Eq.~\eqref{binder4}). We also plot the peak values of the specific heat curve ($T_{C}$) and the peak in the susceptibility of ice rule order parameter $T_{\chi_\rho}$. For the AFM case, the corresponding values are given in Table~\ref{table}. Clearly, the values the gap between $T_c$ and $T_C$ (or $T_{\chi_\rho}$) is decreasing with the strength of $J_3$.\\
\indent \textit{Ferromagnet coupling:} In Fig.~\ref{phase_diagram} (bottom row), we plot the critical points in the $T-K$ plane as obtained from the the Binder cumulant (see Eq.~\eqref{binder4}). We also plot the peak values of the specific heat curve ($T_{C}$) and the peak in the susceptibility of ice rule order parameter $T_{\chi_\rho}$. For the AFM case, the corresponding values are given in Table~\ref{table}. Clearly, the values the gap between $T_c$ and $T_C$ (or $T_{\chi_\rho}$) is increasing with the strength of $J_3$.
%\begin{figure*}[t!]
%\centering
%\includegraphics[width=1.0\textwidth]{Figures/Values_T.pdf}
%\caption{Table showing the critical points ($T_c$) and the points where the specific heat $C$ and the susceptibility $T_{\chi_\rho}$ peaks. (a) For $J_3=0.1$ value of FM type coupling, (b) for $J_3=0.4$, (c) for $J_3=0.8$. }
%\label{table}
%\end{figure*}
%%%%%%%%%%%%%%%%%%%%%%%%%%%%%%%%%%%%%%%%%%%%%%%%%%%%%%%%
\subsection{Dynamic structure factor}
\label{dynamic}
\indent \indent As mentioned, an earlier study that has not accounted for the interlayer coupling has shown three distinct phases as mentioned~\cite{ishizuka2011quantum,vijigiri2020dipole}. Here, we numerically perform simple linear-spin wave dynamic structure calculations on a real space lattice over the classical ground states of the respective phases identified in a quasi-2D version. We do it on a quasi-2D structure because the interlayer coupling is shown to alter the critical points where the qualitative behavior is not changed. So, the dynamic structure factor can be qualitatively extrapolated from quasi-2D results when an interlayer coupling is introduced for the 3D case. Nevertheless, we note that the present study offers scope to get the low-lying excitations (that can be experimentally tractable) above the classical ground states~\cite{vijigiri2018classical}. Crucial information regarding the degree of fluctuations in the ice-rule physics could also be the best probed to validate certain underlying theories~\cite{chern2014gauge}. We know that in organic ferroelectrics each symmetric arrangement of the molecule posses an associated vibrational mode with it. We, therefore, expect that the amount of frustration mandated by its symmetry can be observed experimentally in the non-resonant Raman-scattering. \\
\indent The brief details on the method and implementation of RS-LSWT are given in Appendix~\ref{RSLSWT}. Fig.~\ref{density_plot} shows the $xx$ dynamical correlation function $S^{xx}(k,\omega)$ plotted for three different regimes under small magnetic fields. In Fig.~\ref{density_plot}(a), we see two bands, one of them is a flat one (no dispersion, see Fig.~\ref{density_plot}(a) at $\omega\sim 0.36(7)$) and the other with a dispersion. The flat band can be understood to be the excitations corresponding to the conserved quantities with a non-local gauge symmetry. Within the linear-spin wave theory probing the excitations under the classical ground states, the gauge operator can be decomposed into non-interacting pairs of correlators, for example, $\sigma^z_1\sigma^z_2\sigma^z_3\sigma^z_4$ $\sim $ $\sigma^{x^\prime}_1\sigma^{x^\prime}_2+\sigma^{x^\prime}_3\sigma^{x^\prime}_4+\cdots$ in the transformed coordinates. Since, for $J_2=0$, the terms $\sigma^x_1\sigma^x_2$ will yield a dispersive term like $\cos k$ and since the further configurations along the unit vectors are independent, we see a $\sin k$ behavior is also likely possible. Therefore, ending with the net result being non-dispersive. Even if one considers a mean-field theory extension to the present scenario we might still see a similar flat band, because these terms do not vanish. Note that the minimum of the spectrum is at multiple $k$-points, i.e, $\omega\sim 0.28$ from $M$ to $\Gamma$ points. This could also reflect the underlying symmetry in the system.\\
\indent Similarly, the optical mode that we see from $\Gamma$ to $X$ and $X$ to $M$ in Fig.~\ref{density_plot}(a) could be associated to the excitations of the correlators of the form $\sigma^{x^\prime}_1\sigma^{x^\prime}_2+\sigma^{x^\prime}_3\sigma^{x^\prime}_4+\cdots$ coming from the gauge operators and as well as the intramolecular interaction term. The decomposition can be understood like all the possible dimer coverings on a plaquette spanning the entire lattice. So then the excitations should correspond to these dimers. Now the reason for the appearance of the optical mode can be understood from ice-rule constraint. Since the configurations along the unit vector that spans the lattice are not completely independent. We expect that the net superposition of the spin-waves should be non-dispersive. That is why we see optical modes when only one component is varied ($k_x$ from $\Gamma$ to $X$ or $k_y$ from $X$ to $M$). But when both the components are varied we see that the net result is flat (from $M$ to $\Gamma$). When $J_2$ is turned on, due to the global ordering being established, we see that the excitations of these dimer coverings have a definite spectrum of the form $\cos k_x$ or $\sin k_y$. We thus see no flat band behavior in Fig.~\ref{density_plot}(a). \\
\section{Discussions and summary}
\label{discussion}
\indent \indent We have simulated a three-dimensional model of squaric acid crystal considering the interlayer interaction ($J_3$) of protons that were not rigorously accounted earlier. We use the pseudo-spin formalism and introduce an additional interlayer interaction ($J_3$) of the Ising type. We aim at understanding the role played by the nature of the $J_3$ and the strength of it on the phases that have been found for quasi-2D case. Motivated by earlier works~\cite{matsushita1982cluster}, in the present work, we use three different types of interlayer coupling (AFM, FM, Disorder). Our QMC analysis reveals interesting results for small $J_2$ strengths. Similar to the quasi-2D case~\cite{ishizuka2011quantum,vijigiri2020dipole}, we observe an intermediate-state appearing (around $T\sim 0.3$) where the order parameter, $P$, is found to exhibit a plateau behavior that vanishes with the system size. This is not the scenario for all the types of couplings considered. From Fig.~\ref{ice_param}(a), for an AFM case of interlayer coupling, we see that the AFE stacked configuration of layers seems to be more robust (see Fig.~\ref{order_param}) as the value of $P$($\sim 1$) at low temperatures does not change even when $J_3$ is increased. Since the quantum fluctuations in the AFM case has more pronounced quantum fluctuations, therefore, the intermediate liquid-like states is found to vanish as the strength of $J_3$ is increased (see Fig.~\ref{ice_param}. The plateau starts getting narrowed and then disappears for large $J_3$). Secondly, when we consider the FM type of $J_3$, we observe that not only does the AFE stacked configuration is found to exist for large $J_3$ ($J_3<1.0$), but also the intermediate state was shown to extend to larger temperatures (and fields). In contrast to the AFM coupling of $J_3$, the anomalous peak seen at lower temperatures in the specific-heat curves is shown to have lesser entropy. As the peak height is lesser than in the case of AFM. We know that the entropy ($S\propto \ln(C/T) $). Therefore, the higher the peak higher the more degeneracy the system possesses. Nevertheless, in the last case of disordered coupling, we see neither of the AFE nor intermediate liquid-like state is shown to exist beyond a disorder strength of $J_3=0.4$. This is intuitive because the disorder case accounts for a large degree of quantum fluctuations compared to AFM and FM resulting in such behavior. \\  
\indent  We note that previous studies~\cite{vijigiri2020dipole,ishizuka2011quantum} has investigated the finite-temperature phase diagram in a related model with two body interactions~\cite{ishizuka2011quantum} followed by a four-spin model~\cite{vijigiri2020dipole,chern2014gauge} motivated by the earlier studies~\cite{maier1982interacting}. Though it is known that the interlayer interaction is a weak one, some studies have reported that in $\text{H}_2\text{SQ}$ interlayer interaction is more likely AFM in nature. Here, in the present work, we show that AFM type $J_3$ has a more robust anomalous peak compared to the others. And the stability of the intermediate state strictly depends on the strength. While this is the scenario for AFM, the FM coupling also hosts the AFE phase, but with the exception that unlike the AFM coupling the intermediate state extends to larger temperatures and pressures as the strength of $J_3$ is increased. \\
\indent Further, apart from the ground state properties and static susceptibilities, to probe the nature of excitations that can be experimentally verified, we also calculate the dynamic structure factor in our calculations. We use a real-space linear spin-wave theory to obtain the spectrum. We see that the characteristic behavior of the deconfined phase can be seen in Fig.~\ref{density_plot}(a) when $J_{1,2}$ are absent. That is, apart from the flat spectrum in Fig.~\ref{density_plot}(a) the spectrum is highly asymmetric around the $X, M$ high symmetry point. This is expected because in the deconfined phase the system along the two orthogonal directions has an independent configuration. The configuration along the unit vector $x$ is independent of configuration along $y$. Due to this fact, we see that the spectrum until the $M-$high symmetry point is broader while the spectrum from $M$ to $\Gamma$ is flat. This explains the deconfined characteristic. Next, once we put the finite $J_1$ with small $J_2$, we see that a small deviation from the flat spectrum from $M$ to $\Gamma$ is seen. This is because the small $J_2$ is responsible for a global ordering in the system. And in the global ordering case, the ordering along the two orthogonal vectors spanning the lattice is no more independent. This can be further corroborated from the Fig.~\ref{density_plot}(c) when the $J_2$ is further increased a sinusoidal behavior in the spectrum is seen. We know that any ordering in the system manifests in the form of $\sim \cos(\vec{k})$ or $\sin(\vec{k})$. This explains the dynamic structure factor in three important regimes. 
\section*{Acknowledgment}
All the computations performed in this work were part of the Konark High-performance computing facility supported by the Institute of physics.
\section*{Appendix}
%\subsection{pCUT}
%(\mandal{I see that the results on this one particle dispersion have been removed.  can one formulate it for some very small values of $J_3$ for $J_2=0$? Then also it portends some three-dimensional results and hence can be included.})
%Here, we aim to obtain the equivalence of the current Hamiltonian ~\ref{Hamiltonian} to that of the Toric-Code Model~\cite{vidal2009} in the absence of dipole-dipole interaction, $J_2$ and interlayer coupling $J_3$. We show below that the physics of the current model can be exactly described by the TCM under $K_x\neq 0$, $K_{y,z}=0$. Interestingly, the equivalence can only be established for specific conditions that $J_0=J_1$, rest of the cases, this may not be the scenario. \\
%\indent In the low field limit $J_{0,1} \ggg K$, for 2D intraplane structure in the absence of dipole-dipole interaction, we calculate the one-particle dispersion within the perturbative Continuous Unitary Transformation(pCUT) formalism. Here we shall outline the implementation details of pCUT and give the hopping amplitudes and the corresponding ground state energy up to order 6.
%From the earlier studies we know that the low-field case hosts a deconfined phase with gauge charge excitations coming from the gauge term $J_0$ similar to the excitations in the Toric-code model(TCM) except the fact that in TCM there are two kinds of excitations one is the magnetic charge and the other being the gauge charge corresponding to two types of interactions. It should be noted that the true one-particle excitation on a lattice can be achieved only in the open boundary conditions where one can separate the excitation pairs ideally without any energy cost to infinite distance effectively becoming a one-particle sector.  However, It is to be noted that even in the absence of dipole-dipole interaction the spectrum of unperturbed Hamiltonian isn't equidistant, and to make pCUT applicable we, therefore, choose the parameters so that within the set parameters it isn't the case anymore.\\
%\begin{align}
%e_1&=1 + K (-2 \cos(q_x - q_y) - 2 \cos(q_x + q_y)) + \\ \nonumber
% & K^2 (2 - 2 \cos(2 q_x) - \cos(q_x - q_y) - 2 \cos(2 q_y)- \\ \nonumber
% & \cos(q_x + q_y)) + \\ \nonumber
% & K^3 (-3 \cos(q_x - 3 q_y) - 3 \cos(3 q_x - q_y)  - 3 \cos(3 q_x + q_y) \\ \nonumber 
% & - 3 \cos(q_x + 3 q_y)) \\ \nonumber 
%    & K^4 (\frac{35}{2} - 2 \cos(2 q_x) - \frac{15}{2} \cos(4 q_x)   - 5 \cos(q_x - 3 q_y) % \\ \nonumber
% &- \frac{25}{4} \cos(2 q_x - 2 q_y) - \cos(q_x - q_y) -  5 \cos(3 q_x - q_y) - \\ %\nonumber 
%   & 2 \cos(2 q_y) - \frac{15}{2} \cos(4 q_y) - \cos(q_x + q_y) \\ \nonumber & - 5 \cos(3 %q_x + q_y) - 
%    \frac{25}{4} \cos(2 q_x + 2 q_y) - 5 \cos(q_x + 3 q_y))  \\ \nonumber  
%\end{align}
%\indent The configurational energy sets for a single given plaquette(four spins) in the absence of magnetic field $K$ are $-J_0 + 2J_1$, $-J_0+2J_1$, $J_0$, Clearly for two cases the spectrum is equidistant, i.e, for $J_1=0$ and $J_1=J_0$ with corresponding $-3J_0$, $J_0$ sets. One may shift the energy to the energies or renormalize it to give the energies $-J_0$ and $J_0$ respectively. Thus we finally get an equidistant unperturbed spectrum. We use a single period cluster lattice to obtain the physical properties such that the order of perturbative expansion, $n$,  doesn't account finite-size effects. In other words, $L \ge n+2$. \\
%\indent The ground state energy per spin $\text{e}_{\text{GS}}$ up to order 6 is given by,
%\begin{eqnarray}
%e_{GS}=-J_0-2J_1 - \frac{1}{2}K^2 - \frac{15}{8}K^4 - \frac{147}{8}K^6
%\end{eqnarray}
%and similarly we give the one particle dispersion $e_{1}$ up to order 4, order 5 and 6 are too lengthy to be given here but will be effectively given when the one particle gap is obtained at the minimum K-point, i.e, $\Gamma$. The $\Delta$ is given by,
\subsection{Real space Linear-Spin wave theory}
\label{RSLSWT}
Following Ref.~\cite{moessner2019} the dynamic structure factor was obtained within the semi-classical approach of Linear-spin wave theory calculated in the real space. Below is a brief description of the method and the implementation details of our code. Since we know the classical ground states obtained from the previous studies We, therefore, perform the Holstein-Primakoff transformation upon these ground states, neglecting terms beyond quadratic in $a,a^{\dagger}$ we diagonalize the Hamiltonian in the basis of $a,a^{\dagger}$ in the real-space. Then using the dynamical matrix method approach the obtained eigenvalues($\omega_\alpha$) and eigenvectors($\psi^{i}_{\alpha}$) are used to calculate the DSF as,
\begin{align}
 S^{xx}(\kappa,\omega)&= \frac{1}{2N}\sum_{i,j=1}^{N}e^{-ik\cdot(r_i - r_j)}\\ \nonumber
 & \times \sum_{\alpha}(\eta_{i}^{x}\cdot\psi_{\alpha}^{i})(\eta_{j}^{x}\cdot\psi_{\alpha}^{j})^{\star}\delta(\omega-\omega_\alpha)\text{sgn}(\omega_\alpha)
\end{align}
where $\eta_{\alpha}^{i}$ are the matrix elements of the rotational matrix obtained after projecting the local $z$-projected classical ground states onto the global coordinate frame. Since at the quadratic level most of the ground states haves same Hamiltonian elements and hence the averaging of the structure factor here is not necessary. The results in sec~\ref{dynamic} are calculated on a 16*16 lattice with periodic boundary conditions invoked. 
%\bibliographystyle{aipauth4-1}
\bibliography{manus-3rd-ref}
%\printbibliography
\end{document}
